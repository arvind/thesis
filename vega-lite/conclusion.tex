% !TEX root = ../thesis.tex
\section{Conclusion}
\label{sec:vl:conclusion}

To our knowledge, Vega-Lite is the first high-level visualization language to
offer a multi-view grammar of graphics tightly integrated with a grammar of
interaction. The resulting concise specifications facilitate rapid exploration
of design variations.

An early version of Vega-Lite has already been well-received by the broader
community. Third-party bindings have been created for a number of environments
including Python~\cite{vega-lite:altair}, R~\cite{vega-lite:hrbrmstr,
vega-lite:timelyportfolio}, Scala~\cite{vega-lite:scala},
Julia~\cite{vega-lite:julia}, and a REPL client for
Clojure~\cite{vega-lite:clojure}. In a widely-shared review of Python
visualization libraries, community member Dan Saber noted that \emph{``it is
this type of 1:1:1 mapping between thinking, code, and visualization that is my
favourite thing about [Vega-Lite].''} Moreover, members of the Jupyter team have
called Vega and Vega-Lite \emph{``perhaps the best existing candidates for a
principled \emph{lingua franca} of data visualization.''}

Vega-Lite has also had an impact in the research community. It has been used to
reverse-engineer visualizations from chart images~\cite{poco:reverse}, build a
model for sequencing visualizations~\cite{kim:graphscape}, and powers the
CompassQL recommendation engine~\cite{voyager, compassql}. Such work is
possible, in part, due to well-tested \emph{effectiveness
criteria}~\cite{bertin:semiology, cleveland:perception, mackinlay:apt} for
visual encodings. One promising avenue for future work is to use Vega-Lite to
derive analogous criteria for interaction techniques.

Vega-Lite is an open source system available at
\url{http://vega.github.io/vega-lite/}. We hope that it enables analysts to
produce and modify interactive graphics with the same ease with which they
currently construct static plots.