% !TEX root = ../thesis.tex
\section{Limitations \& Future Work}
\label{sec:vl:conclusion}

The previous section demonstrates that Vega-Lite specifications are more concise
than those of the lower-level Vega language, and yet are sufficiently expressive
to cover an interactive visualization taxonomy. Moreover, we have shown how
primitives can be systematically enumerated to facilitate exploration of
alternative designs. Nevertheless, we identify two classes of limitations that
currently exist.

First, there are limitations that are a result of how our formal model has been
reified in the current Vega-Lite implementation. In particular, components that
are determined at compile-time cannot be interactively manipulated. For example,
a selection cannot specify alternate fields to bin or aggregate over. Similarly,
more complex selection types (e.g., lasso selections) cannot be expressed as the
Vega-Lite system does not support arbitrary path marks. Such limitations can be
addressed with future versions of Vega-Lite, or alternate systems that
instantiate its grammar. For example, rather than a \emph{compiler},
interactions could parameterize the entirety of a specification within a
Vega-Lite \emph{interpreter}.

The second class of limitations are inherent to the model itself. As a
higher-level grammar, our model favors conciseness over expressivity. The
available primitives ensure that common methods can be rapidly specified, with
sufficient composition to enable more custom behaviors as well. However, highly
specialized techniques, such as querying time-series data via relaxed
selections~\cite{holz:relaxed}, cannot be expressed by default. Fortunately, our
formulation of selections, which decouple backing points from selected points
via a predicate function, provide a useful abstraction for extending our base
semantics with new, custom transforms. For example, the aforementioned technique
could be encapsulated in a \emph{relax} transform applicable to multi
selections.

While our selection abstraction supports \emph{interactive} linking of marks,
our view algebra does not yet provide means of \emph{visually} linking marks
across views (e.g., as in the Domino system~\cite{gratzl:domino}). Our view
algebra might be extended with support for connecting corresponding marks. For
example, points in repeated dot plots could be visually linked using line
segments to produce a parallel coordinates display.

An early version of Vega-Lite is used to automatically recommend static plots as
part of the Voyager browser~\cite{voyager}. Voyager leverages perceptual
\emph{effectiveness criteria}~\cite{bertin:semiology, cleveland:perception,
mackinlay:apt} to rank candidate visual encodings. One promising avenue for
future work is to develop models and techniques to analogously recommend
suitable interaction methods for given visualizations and underlying data types.
Here, the human-computer interaction literature~\cite{beaudouin:instrumental}
may offer insights on when and why users benefit from certain interaction
techniques.

Vega-Lite is an open source system available at
\url{http://vega.github.io/vega-lite/}. By offering a multi-view grammar of
graphics tightly integrated with a grammar of interaction, Vega-Lite facilitates
rapid exploration of design variations. Ultimately, we hope that it enables
analysts to produce and modify interactive graphics with the same ease with
which they currently construct static plots.