% !TEX root = ../thesis.tex
\section{Compilation}
\label{sec:vl:compiler}

The Vega-Lite compiler ingests a JSON specification and outputs a lower-level
Reactive Vega specification (also expressed as JSON). However, there is no
one-to-one correspondence between components of the Vega-Lite and Vega
specifications. For instance, the compiler has to synthesize a single Vega data
source, with transforms for binning and aggregation, from multiple Vega-Lite
encoding definitions. Conversely, for a single definition of a Vega-Lite
selection, the compiler might generate multiple Vega signals, data sources, and
even parameterize scale extents. Moreover, to facilitate rapid authoring of
visualizations, Vega-Lite specifications omit lower-level details including
scale types and the properties of the visual elements such as the font size. The
compiler must resolve the resulting ambiguities.

To overcome these challenges, the compiler moves through four phases:

\begin{enumerate}
  \item \emph{Parse}\,---\,the compiler parses and disambiguates an input
  specification. Hand-crafted rules are applied to produce perceptually
  effective visualizations. For example, if the color channel is mapped to an
nominal field, and the user has not specified a scale domain, a categorical
color palette is inferred. If the color is mapped to a quantitative field, a
sequential color palette is chosen instead.

  \item \emph{Build}\,---\,the compiler builds an internal representation to map
  between Vega-Lite and Vega primitives. A tree of \emph{models} is constructed;
  each model corresponds to a unit or composite view, and stores a series of
  \emph{components}. Components are data structures that loosely correspond to
  Vega primitives (such as data sources, scales, and marks). For example, the
  \texttt{DataComponent} details how the dataset should be loaded (e.g., is it
  embedded directly in the specification, or should it be loaded from a URL, and
  in what format), which fields should be aggregated or binned, and what filters
  and calculations should be performed.

  In this step, compile-time selection transforms (those not parameterized by
  events) are applied to the requisite components. For example, the
  \emph{project} transform overrides the \texttt{SelectionComponent}'s predicate
  function, while the \emph{nearest} transform augments the
  \texttt{MarkComponent} with a Voronoi diagram. This phase also constructs a
  special \texttt{LayoutComponent} to calculate suitable spatial dimensions for
  views. This component emits Vega data sources and transforms to calculate a
  bottom-up view layout at runtime.

  \item \emph{Merge}\,---\,once the necessary components have been built, the
  compiler performs a bottom-up traversal of the model tree to merge redundant
  components. This step is critical for ensuring that the resultant Vega
  specification does not perform unnecessary computation that might hinder
  interactive performance. To determine whether components can be merged, the
  compiler computes a hash code and compares components of the same
  type. For example, when a scatterplot matrix is specified using the
  \emph{repeat} operator, merging ensures that we only produce one scale for
  each row and column rather than two scales per cell ($2N$ versus $2N^2$
  scales). Merging may introduce additional components if doing so results in a
  more optimal representation. For example, if multiple units within a composite
  specification load data from the same URL, a new \texttt{DataComponent} is
  created to load the data and the units are updated to inherit from it instead.
  This step also unions scale domains and resolves \texttt{SelectionComponents}.

  \item \emph{Assemble}\,---\,the final phase assembles the requisite Vega
  specification. In particular, \texttt{SelectionComponents} produce signals to
  capture events and the necessary backing points, and multi and interval
  selections construct data sources as well to hold multiple points. Each
  run-time selection transform (i.e., those that are triggered by an event)
  generates signals as well, and may augment the selection's data source with
  data transformations. For example, the \emph{translate} transform adds a
  signal to capture an ``anchor'' position, to determine where panning begins,
  and another to calculate a ``delta'' from the anchor. These two signals then
  feed transforms that offset the backing points stored in the selection's data
  source, thereby moving the brush or panning the scales.
\end{enumerate}
