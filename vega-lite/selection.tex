% !TEX root = ../thesis.tex
\section{Interactive Selections}
\label{sec:vlgoi}

To support specification of interaction techniques, we extend the definition of
unit specifications to also include a set of \emph{selections}. Selections
identify the set of points a user is interested in manipulating, and is formally
defined as an eight-tuple:

\centerline{
  \emph{selection := (name, type, predicate, domain$|$range, event, init,
  transforms, resolve})
}

When an input \emph{event} occurs, the selection is populated with \emph{backing
points} of interest. These points are the minimal set needed to identify all
\emph{selected points}. The selection \emph{type} determines how many backing
values are stored, and how the \emph{predicate} function uses them to determine
the set of selected points. Supported types include a \emph{single} point,
\emph{multiple} discrete points, or a continuous \emph{interval} of points.

As its name suggests, a single selection is backed by one datum, and its
predicate tests for an exact match against properties of this datum. It can also
function like a dynamic variable (or \emph{signal} in
Vega~\cite{reactive-vega-model}), and can be invoked as such. For example, it
can be referenced by name within a filter expression, or its values used
directly for particular encoding channels. \emph{Multi} selections, on the other
hand, are backed by datasets into which data values are inserted, modified or
removed as events fire. They express discrete selections, as their predicates
test for an exact match with at least one value in the backing dataset. The
order of points in a multi selection can be semantically meaningful, for example
when a multi selection serves as an ordinal scale domain.
\todo{PointSelectionFig} illustrates how points are highlighted in a scatterplot
using point and list selections.

Intervals are similar to multi selections. They are backed by datasets, but
their predicates determine whether an argument falls within the minimum and
maximum extent defined by the backing points. Thus, they express continuous
selections. The compiler automatically adds a rectangle mark, as shown in
\todo{IntervalSelections(a)}, to depict the selected interval. Users can
customize the appearance of this mark via the \texttt{mark} keyword, or disable
it altogether when defining the selection.

Predicate functions enable a minimal set of backing points to represent the full
space of selected points. For example, with predicates, an interval selection
need only be backed by two points: the minimum and maximum values of the
interval. While selection types provide default definitions, predicates can be
customized to concisely specify an expressive space of selections. For example,
a single selection with a custom predicate of the form
\texttt{datum.binned\_price == selection.binned\_price} is sufficient for
selecting all data points that fall within a given bin.

By default, backing points lie in the data \emph{domain}. For example, if the
user clicks a mark instance, the underlying data tuple is added to the
selection. If no tuple is available, event properties are passed through inverse
scale transforms. For example, as the user moves their mouse within the data
rectangle, the mouse position is inverted through the \texttt{x} and \texttt{y}
scales and stored in the selection. Defining selections over data values, rather
than visual properties, facilitates reuse across distinct views; each view may
have different encodings specified, but are likely to share the same data
domain. However, some interactions are inherently about manipulating visual
properties\,---\,for example, interactively selecting the colors of a heatmap.
For such cases, users can define selections over the visual \emph{range}
instead. When input events occur, visual elements or event properties are then
stored.

The particular events that update a selection are determined by the platform a
Vega-Lite specification is compiled on, and the input modalities it
supports. By default we use mouse events on desktops, and touch events on mobile
and tablet devices. A user can specify alternate events using Vega's event
selector syntax~\cite{reactive-vega-model}. For example,
\todo{PointSelections(c)} demonstrates how \texttt{mouseover} events are used to
populate a list selection. With the event selector syntax, multiple events are
specified using a comma (e.g., \texttt{mousedown, mouseup} adds items to the
selection when either event occurs). A sequence of events is denoted with the
between-filter. For example, \texttt{[mousedown, mouseup] > mousemove} selects
all \texttt{mousemove} events that occur between a \texttt{mousedown} and a
\texttt{mouseup} (otherwise known as ``drag'' events). Events can also be
filtered using square brackets (e.g., \texttt{mousemove [event.pageY > 5]} for
events at the top of the page) and throttled using braces (e.g.,
\texttt{mousemove\{100ms\}} populates a selection at most every 100
milliseconds).

\subsection{Selection Transforms}

Analogous to data transforms, selection transforms manipulate the components of
the selection they are applied to. For example, they may perform operations on
the backing points, alter a selection's predicate function, or modify the input
events that update the selection. Unlike data transforms, however, specifying an
ordering to selection transforms is not necessary as the compilation step
ensures commutativity. All transforms are first parsed, setting properties on an
internal representation of a selection, before they are compiled to produce
event handling and interaction logic.

We identify the following transforms as a minimal set to support both common and
custom interaction techniques. Additional transforms can be defined and
registered with the system, and then invoked within the specification. In this
way, the Vega-Lite language remains concise while ensuring extensibility for
custom behaviours.

% Transforms may be composed\,---\,for
% example, the \emph{toggle} and \emph{nearest} transforms can be applied to a
% multi selection in order to toggle membership of the point nearest the user's
% cursor.

\subsubsection{Project}

\centerline{\emph{project(fields, channels)}}

The \emph{project} transform alters a selection's predicate function to
determine inclusion by matching only the given \emph{fields}. Some fields,
however, may be difficult for users to address directly (e.g., new fields
introduced due to inline binning or aggregation transformations). For such
cases, a list of \emph{channels} may also be specified (e.g., \texttt{color},
\texttt{size}). \todo{PointSelections(d, e)} demonstrate how \emph{project} can
be used to select all points with matching \texttt{Origin} fields, for example.
This transform is also used to restrict interval selections to a particular
dimension (\todo{IntervalSelections(c)}).

\subsubsection{Toggle}

\centerline{\emph{toggle(event)}}

The \emph{toggle} transform is automatically instantiated for uninitialized
multi selections. When the \emph{event} occurs, the corresponding data value is
added or removed from the multi selection's backing dataset. By default, the
toggle \emph{event} corresponds to the selection's triggering event, but with
the shift key pressed. For example, in \todo{PointSelections(b)}, additional
points are added to the list selection on shift-click (where \texttt{click} is
the default event for list selections). The selection in
\todo{PointSelections(c)}, however, specifies a custom \texttt{mouseover} event.
Thus, additional points are inserted when the shift key is pressed and the mouse
cursor hovers over a point.

\subsubsection{Bind}

\centerline{\emph{bind(widgets$|$scales)}}

The \emph{bind} transform establishes a two-way binding between control widgets
(e.g., sliders, textboxes, etc.) or scale functions for single and interval
selections respectively.

When a single selection is bound to query widgets, one widget per projected
field is generated and may be used to manipulate the corresponding predicate
clause. When triggering events occur to update the selected points, the widgets
are updated as well. Control widgets, in addition to direct manipulation
interaction, allow for more rapid and exhaustive querying of the backing
data~\cite{shneiderman:dynamicqueries}. For example, scrubbing a slider back and
forth can quickly reveal a trend in the data or highlight a small number of
selected points that would otherwise be difficult to pick out directly.

Interval selections can be bound to the scales of the unit specification they
are defined in. Doing so \emph{initializes} the selection, populating it with
the given scales' domain or range, and parameterizes the scales to use the
selection instead. Binding selections to scales allows scale extents to be
interactively manipulated, yet remain automatically initialized by the input
data. By default, both the \texttt{x} and \texttt{y} scales are bound; alternate
scales are specified by \emph{projecting} over the corresponding channels.

\subsubsection{Translate}

\centerline{\emph{translate(events, by)}}

The \emph{translate} transform offsets the spatial properties (or corresponding
data fields) of backing points by an amount determined by the coordinates of the
sequenced \emph{events}. For example, on the desktop, drag events
(\texttt{[mousedown, mouseup] > mousemove}) are used and the offset corresponds
to the difference between where the \texttt{mousedown} and subsequent
\texttt{mousemove} events occur. If no coordinates are available (e.g., as with
keyboard events), a \emph{by} argument should be specified. This transform
respects the \emph{project} transform as well, restricting movement to the
specified dimensions. This transform is automatically instantiated for interval
transforms, enabling movement of brushed regions (\todo{IntervalSelections(b)})
or panning of the visualization when bound to scale functions
(\todo{PanZoomGrid}).

\subsubsection{Zoom}

\centerline{\emph{zoom(event, factor)}}

The \emph{zoom} transform applies a scale factor, determined by the \emph{event}
to the spatial properties (or corresponding data fields) of backing points. A
\emph{factor} must be specified if it cannot be determined from the events
(e.g., when arrow keys are pressed). As with the \emph{translate} transform, the
\emph{project} transform is respected, allowing for single-dimensional zooming.

\subsubsection{Nearest}

\centerline{\emph{nearest()}}

The \emph{nearest} transform computes a Voronoi decomposition, and augments the
selection's event processing. The data value or visual element nearest the
triggering \emph{event} is now selected (approximating a Bubble
Cursor~\cite{grossman:bubble}). Currently, the centroid of each mark instance is
used to calculate the Voronoi diagram but we plan to extend this operator to
account for boundary points as well (e.g., rectangle vertices).

\subsection{Selection-Driven Visual Encodings}

Once selections are defined, they parameterize visual encodings to make them
interactive\,---\,visual encodings are automatically reevaluated as selections
change. First, selections can be used to drive \emph{conditional} encoding
rules. Each data tuple participating in the encoding is evaluated against the
selection's predicate, and properties are set based on whether it belongs to the
selection or not. For example, as shown in \todo{PointSelections}, the fill
color of the scatterplot circles is determined by a data field if they fall
within the \texttt{id} selection, or set to grey otherwise.

Next, selected points can be explicitly materialized and used as input data for
other encodings within the specification. By default, this applies a selection's
predicate against the data tuples (or visual elements) of the unit specification
it is defined in. To materialize a selection against an arbitrary dataset, a
\emph{map} transform rewrites the predicate function to account for differing
schemas. Using selections in this way enables linked interactions, including
displaying tooltips or labels, and cross-filtering.

Besides serving as input data, a materialized selection can also define scale
extents. Initializing a selection with scale extents offers a concise way of
specifying this behavior within the same unit specification. For multi-view
displays, selection names can be specified as the domain or range of a
particular channel's scale. Doing so constructs interactions that manipulate
viewports, including panning \& zooming (\todo{PanZoomGrid}) and
overview\,+\,detail (\todo{ODIndexChart(a)}).

In all three cases, selections can be composed using logical \texttt{OR},
\texttt{AND}, and \texttt{NOT} operators. As previously discussed, single
selections offer an additional mechanism for parameterizing encodings.
Properties of the backing point can be directly referenced within  the
specification, for example as part of a filter or calculate expression, or to
determine a visual encoding channel without the overhead of a conditional rule.
For example, the position of the red rule in \todo{ODIndexChart(b)} is set to
the \texttt{date} value of the \texttt{indexPt} selection.

\subsection{Disambiguating Composite Selections}

Selections are defined within unit specifications to provide a default
context\,---\,a selection's events are registered on the unit's mark instances,
and materializing a selection applies its predicate against the unit's input
data by default. When units are composed, however, selection definitions and
applications become ambiguous.

Consider \todo{ResolveSelections(a)}, which illustrates how a scatterplot matrix
(SPLOM) is constructed by repeating a unit specification. To brush, we define an
interval selection (\texttt{region}) within the unit, and use it to perform a
linking operation by parameterizing the color of the circle marks. However,
there are several ambiguities within this setup. Is there one \texttt{region}
for the overall visualization, or one per cell? If the latter, which cell's
\texttt{region} should be used to highlight the points?  This ambiguity recurs
when selections serve as input data or scale extents, and when selections share
the same name across a layered or concatenated views.

Several strategies exist for resolving this ambiguity. By default, a
\emph{global} selection exists across all views. With our SPLOM example, this
setting causes only one brush to be populated and shared across all cells. When
the user brushes in a cell, points that fall within it are highlighted, and
previous brushes are removed.

Users can specify an alternate ambiguity resolution when defining a selection.
These schemes all construct one instance of the selection per view, and define
which instances are used in determining inclusion. For example, setting a
selection to resolve to \emph{independent} creates one instance per view, and
each unit uses only its own selection to determine inclusion. With our SPLOM
example, this would produce the interaction shown in \todo{ResolveSelections
(b)}.Each cell would display its own brush, which would determine how only its points
would be highlighted.

Selections can also be resolved to \emph{union} or \emph{intersect}. In these
cases, all instances of a selection are considered in concert: a point falls
within the overall selection if it is included in, respectively, at least one of
the constituents or all of them. More concretely, with the SPLOM example, these
settings would continue to produce one brush per cell, and points would
highlight when they lie within at least one brush (\emph{union}) or if they are
within every brush (\emph{intersect}) as shown in \todo{ResolveSelections(c,
d)}.We also support \emph{union others} and \emph{intersect others} resolutions,
which function like their full counterparts except that a unit's own selection
is not part of the inclusion determination. These latter methods support
cross-filtering interactions, as in Figs.~\ref{fig:SimpleCrossFilter} \&
~\ref{fig:LayeredCrossFilter}, where interactions within a view should not
filter itself.