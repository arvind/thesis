% !TEX root = ../thesis.tex
\section{Implementation Details}

Lyra is a web-based HTML5 application built using
AngularJS\footnote{http://angularjs.org}. Vega is used extensively throughout
the system both to parse data transformations and to represent and generate
visualizations. While close, the mapping between Lyra and Vega abstractions is
not one-to-one. To reduce complexity, Lyra consolidates some Vega mark types. A
line mark in Lyra translates to one of Vega's line, path, or rule marks based on
which properties and fields the designer chooses to map. Similarly, Vega's image
marks are simply rectangle marks in Lyra with an ``image fill.''

Lyra also simplifies Vega's handling of hierarchical data. In a Vega
specification, each level of a hierarchy is assigned to a ``group'' mark, with
children marks inheriting the corresponding subset of data. Lyra insulates users
from this scheme by automatically performing \emph{group injection}: when a user
assigns a mark to a pipeline containing hierarchical data, Lyra automatically
adds and nests the necessary group marks for each level of the hierarchy in the
resulting Vega specification. Group injection makes building small multiples
easy: an exposed \texttt{layout} property lets designers choose between having
groups overlap, layout horizontally, or layout vertically.

Lyra's direct manipulation interactors (handles, connectors, drop zones) are
also generated using Vega, using a separate specification rendered \emph{over}
the visualization. The geometry of the marks serves as input data for this
interactive layer. By decoupling the visualization and interactors, we ensure
that Lyra features do not interfere with the designer's visualization.