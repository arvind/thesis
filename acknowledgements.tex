% !TEX root = ../thesis.tex
\prefacesection{Acknowledgments}
\label{sec:acknowledgements}

By every measure, Jeffrey Heer has been a consummate advisor. Early in my
graduate student career, I was confronted with a fork in the road: remain at
Stanford, a place I had worked hard to get to, or follow Jeff up to the
University of Washington. Though I had to leave behind a significant other and a
fledgling start-up, the choice was clear. It was Jeff's warmth, insight, and
vision that had attracted me to Stanford; if he was moving, so must I! This
dissertation more than validates that decision. Jeff has helped me not only grow
as a researcher but, by frequently diving deep into the Vega code base, hone my
software engineering skills as well. His commitment to producing high-caliber
research \emph{and} ensuring it is released as open-source work remains an
ongoing source of inspiration. If I am half the advisor he is, my future
students will be very luck indeed!

I have also been fortunate to have had the guidance of several other mentors. My
undergraduate advisor, Jim Hollan, fueled my interest in human-computer
interaction. A summer in his lab, where he gave me free reign to explore
questions in multimodal interaction, launched my research career and his
continued counsel to \emph{"never let the urgent drive out the important"} kept
me grounded. Wendy Mackay and Michel Beaudouin-Lafon have been critical for
understanding the broader implications of my work (and for giving me ample
excuses to visit Paris!). I am particularly excited about working with them on
some of the future research directions we brainstormed over drinks or in the
back of their car! Thank you also to Maneesh Agrawala, James Landay, Chris
R\'{e}, and Jeff Hancock for their time, generosity, and sage advice about the
academic job market.

I am also indebted to my incredible collaborators and colleagues at both
Stanford and the University of Washington. Little of the work described in this
thesis would have been possible without the talent and dedication of Kanit
Wongsuphasawat, Dominik Moritz, Jane Hoffswell, and Ryan Russell. The warm
welcome from Danielle Bragg, Caitlin Bonnar, Felicia Cordeiro, and Daniel
Epstein quickly made me feel like a member of the UW community. And, the
48-hour, caffeine-fueled, pre-CHI-deadline Webzeitgeist session with Ranjitha
Kumar, Jerry Talton, Maxine Lim, and Cesar Torres remains one of my favorite
graduate school memories. Thank you also to staff in both departments, including
Jillian Lentz, Jayanthi Subramanian, Monica Niemiec, Diane Rosano, and Andrea
Kuduk, for their patience with my uncountable administrative questions.

My work has been generously supported first by an SAP Stanford Graduate
Fellowship and later by a graduate fellowship from Google. I credit the
widespread adoption of the Reactive Vega stack, in part, to the opportunities we
have had to spread the word. To that end, I am grateful to our external
collaborators including Irene Ros, Lynn Cherny, K. Adam White, Sue Lockwood,
Jake Vanderplas, Brian Granger, and Yuri Astrakhan.

I am fortunate that graduate school has mostly offered me very high highs. But,
for the occasional low low, I am thankful to have been able to commiserate,
rant, and drink with a fantastic group of friends including Marianne Lontoc,
Patrick Mutchler, Rachel Midura, Lauren Howe, Ben Poole, Megan Lin, Austin
Gibbons, and Jessie Schroeder.

Finally, I dedicate this thesis to my family. To my parents, Manju and Satya,
for fostering my interest in computing, often ``forgetting'' I had used up my
alloted screen time for a particular day! Your unconditional love gave me the
support I needed to forge an identity and career half a world away. And to the
love of my life, Maeva Fincker, for being my rock these past four years. I am so
very lucky to have you on my team, to celebrate my successes, and to confide in
you my fears.

Thank you.