% !TEX root = ../thesis.tex
\section{Conclusion \& Future Work}
\label{sec:vg:conclusion}

Declarative languages are a popular means of authoring
visualizations~\cite{bostock:protovis, bostock:d3, heer:protovisjava}, but have
lacked first-class support for interaction design. In response, we contribute
Reactive Vega, the first language and system architecture to support declarative
visualization and interaction design in a comprehensive and performant fashion.

It is important to note that although Reactive Vega provides an complete
end\--to\--end system\,---\,whereby users invoke the parser to traverse an
input declarative specification and instantiate the necessary architecture
components to render a visualization\,---\,this process can be decoupled.
Reactive Vega's declarative model can be used to implement extensions to D3, and
higher-level tools can opt to manually construct and connect required dataflow
operators.

By simplifying programmatic generation of visualization, Reactive Vega's
declarative JSON syntax has led to a growing ecosystem of higher-level
visualization systems. For example, MapD~\cite{mapd:vega} has integrated Vega
with their GPU-powered database; custom SQL queries can be embedded within the
JSON specification, which is dispatched to the backend server, rendered, and
returned to the client as a PNG image. Similarly, Wikipedia, a security-concious
environment where it would be difficult to allow users to write imperative
visualization code, has recently integrated Vega~\cite{mediawiki:graph} to
enable visualization of data embedded in articles.

Still, improved support for authoring and debugging Vega specifications remains
a promising avenue for future work. In recent work, Hoffswell
et~al.~\cite{hoffswell:debugging} developed a ``time-traveling'' debugger for
Reactive Vega specifications and found that first-time Vega users were able to
accurately trace errors through the specification. Further work, particularly by
instrumenting Reactive Vega's dataflow graph to enable inspection and stepping
through changeset propagation could aid in learnability~\cite{guo:tutor}.
Through the development of such tools, we can also assess the accessibility of
the language. Are new users able to learn the declarative interaction model? Can
experts, accustomed to callback-driven programming, quickly transition as well?

Reactive Vega's architecture also offers opportunities to study scalable
visualization design. Interactive visualization of large-scale datasets often
requires offloading computation to server-side architectures. For example,
Nanocubes~\cite{lins:nanocubes} and imMens~\cite{liu:immens} assemble
multi\--dimensional data cubes that can be decomposed into smaller data tiles
and pushed to the client. Such components could be integrated into a dataflow
graph with execution distributed across server and client~\cite{domoritz:dsia}.
For example, as the dataflow graph scheduler is responsible for propagation, it
might anticipate possible user interactions and prefetch data tiles in order to
reduce latency~\cite{battle:prefetch}.

Reactive Vega is an open source system available at
\url{http://vega.github.io/vega/}.