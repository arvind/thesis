% !TEX root = ./thesis.tex
\section{Comparative Performance Benchmarks}
\label{sec:vg:performance}

To evaluate the performance of Reactive Vega against D3~\cite{bostock:d3} and
the origin, non-reactive Vega system (v1.5.0), we use the same setup described
in the previous section.

\subsection{Streaming Visualizations}

Figure~\ref{fig:static_benchmark} shows the average performance of
(non\--interactive) streaming scatter plots, parallel coordinates plots, and
trellis plots. We first measured the average time to initially parse and render
the visualizations. To gauge streaming performance, we next measured the average
time taken to update and re-render upon adding, modifying, or removing 1\% of
tuples. We ran 10 trials per dataset, sized 100--100,000 tuples.

Reactive Vega has the greatest effect with the parallel coordinates plot,
displaying 2x and 4x performance increases over D3 and Vega 1.5, respectively.
This effect is due to each plotted line being built and encoded by its own
dataflow branch. Across the other two examples, and averaging between the Canvas
and SVG renderers, we find that although Reactive Vega takes 1.7x longer to
initialize the visualizations, subsequent streaming operations are 1.9x faster
than D3. Against Vega 1.5, Reactive Vega is again 1.7x slower at initializing
visualizations; streaming updates perform roughly op-par with the Canvas
renderer, but are 2x faster with the SVG renderer.

Slower initialization times for Reactive Vega are to be expected. D3 does not
have to parse and compile a JSON specification, and a streaming dataflow graph
is a more complex execution model, with higher overheads, than batch processing.
However, with streaming visualizations this cost amortizes and performance in
response to data changes becomes more important. In this case, Reactive Vega
makes up the difference in a single update cycle.

\subsection{Interactive Visualizations}

We evaluated the performance of interactive visualizations (measured in terms of
interactive frame rate) using three common examples: brushing \& linking a
scatterplot matrix, a time-series overview+detail visualization, and panning \&
zooming a scatterplot. We chose these examples as they all leverage interactive
behaviors supported by D3, with canonical implementations available for
each\footnote{Brushing \& Linking:
http://bl.ocks.org/mbostock/4063663}\textsuperscript{,}\footnote{Overview +
Detail: http://bl.ocks.org/mbostock/1667367}\textsuperscript{,}\footnote{Pan \&
Zoom: http://bl.ocks.org/mbostock/3892919}. For Reactive Vega, we expressed
these visualizations with a single declarative specification. For D3 and Vega
1.5, we use custom event handling callbacks. The Vega 1.5 callbacks mimic the
behavior of the fragmented reactive approach used in prior
work~\cite{satyanarayan:declarative}. We tested these visualizations with
datasets sized between 100 and 10,000 tuples.

Figure~\ref{fig:interactive_benchmark} shows the results\,---\,on average, and
across both Canvas and SVG renderers, Reactive Vega offers superior interactive
performance to custom D3 and Vega event handling callbacks. This effect
primarily stems from Reactive Vega's unified data model, and is most noticeable
with brushing \& linking a scatterplot matrix and the time-series
overview+detail visualization. In both examples, interactions manipulate only a
subset of all data tuples. With Reactive Vega, only these tuples are processed,
and their corresponding scene graph elements re-encoded and re- rendered. By
comparison, with Vega 1.5's fragmented reactive approach, the entire scene graph
must be reconstructed and rendered in response to changes in input data.