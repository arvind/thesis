% !TEX root = ./thesis.tex
\prefacesection{Abstract}

\vspace{-30pt}

Interactive visualization is an increasingly popular medium for analysis and
communication as it allows readers to engage data in dialog. Hypotheses can be
rapidly generated and evaluated in situ, facilitating an accretive construction
of knowledge and serendipitous discovery. Crafting effective visualizations,
however, remains difficult. Programming toolkits are typically required for
custom visualization design, and impose a significant technical burden on users.
Moreover, existing models of visualization relegate interaction to a
second-class citizen: imperative event handling callbacks that are difficult to
specify, and even harder to reason about.

This thesis introduces the \emph{Reactive Vega stack}: two new declarative
languages that lower the threshold for authoring interactive visualizations, and
enable higher-level reasoning about the interaction design space. \emph{Reactive
Vega} is an expressive, low-level representation that is well-suited for custom,
explanatory visualizations. It shifts the burden of execution from the user to
the underlying streaming dataflow system. \emph{Vega-Lite} builds on Vega to
provide a higher-level grammar for rapidly specifying interactive graphics for
exploratory analysis. Its concise format decomposes interaction design into
semantic units that can be systematically enumerated.

Together, these languages serve as platforms for further research into novel
methods of expressing visualization design, and interactive data analysis
systems. And, critically, they provide a growing and engaged community to study
their use with\,---\,the Wikipedia and Jupyter communities, for instance, have
embraced Vega and Vega-Lite to author interactive visualizations within articles
and data science notebooks, respectively. This thesis explores this nascent
space of higher-level interactive data systems with \emph{Lyra}, a direct
manipulation visualization design tool.