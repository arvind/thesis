% !TEX root = ./thesis.tex
\prefacesection{Abstract}

% \vspace{-30pt}

Interactive visualization is an increasingly popular medium for analysis and
communication as it allows readers to engage data in dialog. Hypotheses can be
rapidly generated and evaluated in situ, facilitating an accretive construction
of knowledge and serendipitous discovery. Crafting effective visualizations,
however, remains difficult. Programming toolkits are typically required for
custom visualization design, and impose a significant technical burden on users.
Moreover, existing models of visualization relegate interaction to a
second-class citizen: imperative event handling callbacks that are difficult to
specify, and even harder to reason about.

This thesis introduces the \emph{Reactive Vega stack}: two new declarative
languages for \emph{interactive} visualization that decouple specification (the
\emph{what}) from execution (the \emph{how}). At the foundation is
\emph{Reactive Vega}, an expressive representation that models user input as
streaming data. Its underlying dataflow runtime handles the complexity of event
propagation and state management, freeing users to focus on interaction design
decisions. \emph{Vega-Lite} builds on Vega with a high-level grammar for rapidly
authoring interactive graphics for exploratory analysis. It provides a concise
specification format that decomposes interaction design into semantic units that
can be systematically enumerated.

Critically, these languages offer \textsc{json} syntaxes to simplify
programmatic generation of interactive visualization and enable novel
interactive data systems. This thesis develops one such system, \emph{Lyra}, a
direct manipulation tool for visualization design. Drag-and-drop operations in
Lyra generate statements in Vega and Vega-Lite, allowing users to author a
diverse range of visualizations without any textual programming.

These systems have been released as open-source projects, have been widely
adopted, and have given rise to an \emph{ecosystem} of interactive visualization
tools. Users can author an exploratory visualization in the Jupyter Notebook,
export it to Lyra via Vega-Lite and add an explanatory annotation layer, and
then embed the resultant Reactive Vega visualization within a Wikipedia article.
As a result, rather than a single monolithic system, the Reactive Vega stack
facilitates development of targeted applications, and allows users to work at
the level of abstraction most suited for the task at hand.