% !TEX root = ./thesis.tex
\chapter{Related Work}
\label{sec:related_work}

This dissertation builds on prior work in visualization systems, programming
languages, streaming databases, and human-computer interaction. In this chapter,
I survey pertinent work in each domain and describe how this thesis extends
them.

\vspace{-10pt}

\section{Visualization Toolkits \& Systems}

\vspace{-10pt}

Chart templates, as found in spreadsheets and online services (e.g., Many
Eyes~\cite{ibm:manyeyes} and Google Fusion Tables), are a common means for
creating visualizations. These \emph{chart typologies} are popular because they
are simple to use: users select from a pre-defined palette of chart types, and
produce recognizable output with only a few clicks. This approach, however, is
closed. Users are restricted to only the available types, and may customize only
a small handful of parameters.

To enable custom visualization design, researchers have developed programming
toolkits that offer a number of visualization components. For instance, the
InfoVis Toolkit~\cite{fekete:ivtk} provides a class hierarchy of visualization
widgets, where new visualizations are created by subclassing existing components
or writing new ones. In contrast, Prefuse~\cite{heer:prefuse} and
Flare~\cite{flare} offer composable operators for data transform, layout, and
encoding. While either approach offers fine-grained control over visual output,
they both constrain expressivity much like chart typologies. Novel designs
require new widgets or operators, and significant software engineering effort.

\subsection{Grammar-Based Visual Encoding}

\vspace{-10pt}

In 1999, Leland Wilkinson introduced an alternative approach with \emph{The
Grammar of Graphics}~\cite{wilkinson:grammar}. Eschewing chart typologies in
favor of a combinatorial building blocks, Wilkinson's grammar yields a more
expressive design space and allows users to rapidly construct visualizations.
Provided abstractions include primitives for data modeling and transformation,
visual encodings defined as mappings between data fields and channels (e.g.,
position, color, and size), and scales and guides (i.e., axes and legends). His
work was quickly followed by the Stanford Polaris system~\cite{stolte:polaris}
(commercialized as Tableau), and Hadley Wickham's popular
ggplot2~\cite{wickham:layered} package implements a variant of this model in
\textsc{r}.

These tools focus on statistical graphics for exploratory analysis and, to that
end, favor concise specification languages. Analysts may omit details such as
scale transforms (e.g., linear or log) or color palettes, which are then filled
in by the grammar using a rule-based system of smart defaults. As a result,
analysts are able to rapidly generate visualizations, analyze them, and continue
with their analysis process.

Through the lower-level Protovis~\cite{bostock:protovis,heer:protovisjava} and
D3~\cite{bostock:d3} languages, Bostock and Heer describe how a similar approach
can enable an expressive space of custom, explanatory graphics as well. They
also articulate the advantages of declarative visualization design: separating
specification from execution facilitates an iterative development process,
simplifies retargeting across platforms, and enables unobtrusive runtime
optimizations~\cite{heer:protovisjava}.

The design of Reactive Vega and Vega-Lite is heavily influenced by these works.
Reactive Vega's visual encoding primitives and implementation pipeline closely
mirror Protovis'. Marks including \emph{arcs}, \emph{areas}, \emph{bars},
\emph{lines}, plotting \emph{symbols} and \emph{text} are processed through
\emph{bind}, \emph{build}, and \emph{evaluate} stages to generate a scene graph.
Protovis' \emph{panel} marks are renamed to \emph{group} marks, and enable the
repeated and nested structures found in small multiple displays. Marks can be
positioned relative to one-another with Vega's \emph{reactive geometry}, which
provides a more unified approach to layout as compared with Protovis'
\emph{anchors}.

Similarly, Vega-Lite represents basic plots using a set of encoding definitions
that map data attributes to visual channels, and may include common data
transformations. Vega-Lite differs most from other high-level grammars in its
approach to multiple view displays. Each of these grammars supports faceting (or
nesting) to construct trellis plots~\cite{becker:trellis} in which each cell
visualizes a different partition of the data. Both Wilkinson's grammar and
Polaris/Tableau achieve this through a \emph{table algebra} over data fields,
which in turn determines spatial subdivisions. Tableau additionally supports the
construction of multi-view dashboards via a different mechanism, with each view
backed by a separate specification. In contrast, Vega-Lite contributes a
\emph{view algebra}: starting with unit specifications that define a single
plot, composite views are expressed using operators for layering, horizontal or
vertical concatenation, faceting, and parameterized repetition. When applicable,
these operators merge scale domains and properly align constituent views.
Disparate views can also be combined into arbitrary dashboards, all within a
unified algebraic model.

\vspace{-10pt}

\subsection{Dataflow Visualization Systems}

\vspace{-10pt}

Dataflow architectures are common in scientific visualization systems, such as
IBM Data Explorer~\cite{lucas:dataflow,abram:ibmdx} and
VTK~\cite{schroeder:vtk}. Developers must manually specify and connect each
required operator into a network, with updates supported in a demand-driven
fashion (e.g., as data is modified) or an event-driven fashion (e.g., in
response to user input). These systems expose fine-grained control over the
dataflow graph. For example, VTK developers can choose to favor memory
efficiency over processing speed, which causes dataflow operators to delete
their output after computation. While Reactive Vega shares some dataflow
strategies with these systems\,---\,for example, using pass-by-reference for
unchanged tuples to reduce memory consumption\,---\,it abstracts such execution
concerns away from the user. The dataflow graph is automatically assembled based
on definitions found in a declarative specification, and optimizations are
transparently performed such that output data is only stored when needed by
downstream operators and shared otherwise.

Within the domain of information visualization, Stencil~\cite{cottam:stencil} is
also grounded in reactive semantics and uses a dataflow model. Like Reactive
Vega, it provides a unified data model where both input data and interaction
events are modeled as first-class streaming data sources. However, Reactive Vega
is more expressive than Stencil in two important ways. First, Stencil does not
provide primitives, beyond event streams, to support interaction design;
Reactive Vega, on the other hand, offers predicates and scale inversions to lift
interactive selections to data queries, and coordinate multiple displays.
Moreover, Reactive Vega's graphical primitives can be arbitrarily nested,
drawing from either hierarchical or distinct data sources. This ability is
critical to concisely specifying small multiples displays, and requires Reactive
Vega's dataflow graph to dynamically rewrite itself at runtime. To the best of
our knowledge, Stencil does not support self-instantiating dataflows.

Improvise~\cite{weaver:improvise} features active variables called ``live
properties,'' which provide a basic form of reactivity\,---\,they may be bound
to control widgets and parameterize a visualization. Using an expression
language, live properties are assembled into a coordination graph to dynamically
evaluate visual encodings and generate views of data. While Improvise and
Reactive Vega share some conceptual underpinnings, Improvise places a higher
burden on users to correctly construct the necessary graph. As Reactive Vega
takes a declarative approach to visualization design, users need only compose
the necessary primitives into a specification; Reactive Vega parses it to build
the corresponding dataflow graph.

\vspace{-10pt}

\subsection{Interactive Visualization Systems}

\vspace{-10pt}

Besides text-based toolkits, researchers have also developed graphical user
interfaces for constructing visualizations.
Polaris/Tableau~\cite{stolte:polaris}, for instance, displays a ``schema'' panel
that lists available data fields. Data groupings and visual encodings are
specified by dragging these fields onto ``shelves'' that array the periphery of
the visualization. Lyra extends this mode of interaction onto the visualization
canvas itself\,---\,property \emph{drop zones} overlay marks during drag
operations, providing a more direct target for data binding operations.

Roth et al.'s SageBrush~\cite{roth:sagebrush} supports similar interactions.
Users can drag-and-drop ``partial prototypes'' for spatial encodings and
``grapheme'' (mark) primitives such as lines and bars. Custom grapheme
properties are manually selected via menus, then exposed as drop-target icons.
Lyra refines the SageBrush model in several ways. Partial prototypes in
SageBrush implicitly define scale transforms and layout; by leveraging Vega and
Vega-Lite, Lyra decouples these components to support richer designs. Lyra's
data pipelines provide an extensible set of data transformations, and different
marks can be driven by unique pipelines and composed. Finally, Lyra eschews
iconic abstractions in favor of direct drop zones for mark properties, without
need of explicit enumeration via menus.

More recently, Bret Victor demonstrated a data-driven drawing
tool~\cite{victor:drawing} (now developed as Apparatus~\cite{apparatus}) that
combines geometric constraints with \emph{imperative} procedures over data.
Alongside an interactive canvas and data table viewer, the tool includes
programming constructs such as lexical scope, looping control flow, and
conditionals. For a basic bar chart, designers must define loops to create and
position each bar. The tool supports purely geometric constructions, rendering
some layouts inexpressible. By building on Vega and Vega-Lite, Lyra takes a
\emph{declarative} approach to design. When a designer associates a mark with a
dataset, one mark instance per datum is automatically produced rather than
requiring explicit instantiation. More advanced layouts (e.g., cartographic
projections, force-directed layout, etc.) are available through Lyra's graphical
data pipeline.

\vspace{-10pt}

\section{Specifying Interactions in Visualizations}

\vspace{-10pt}

Despite the central role of interaction in effective data visualization
\cite{heer:dynamics, pike:interactionscience}, little work has been done to
develop a grammar for specifying interaction techniques. Wilkinson's grammar
includes no notion of interaction. Tableau supports common interaction
techniques, but relies on mechanisms external to the visual encoding grammar.
Early systems like GGobi~\cite{swayne:ggobi} support common techniques as well,
and provide imperative APIs for custom methods. However, such APIs make easy
tasks needlessly complex, burdening developers with learning low-level execution
details. More recent systems, including Protovis, D3, and
VisDock~\cite{choi:visdock}, offer a typology of common techniques that can be
applied to a visualization. Such top-down approaches, however, limit
customization and composition. For example, D3's interactors encapsulate event
processing, making it difficult to combine them if their events conflict (e.g.,
if dragging triggers brushing \emph{and} panning).

\vspace{-10pt}

\subsection{Interactive Selection \& Querying}

\vspace{-7pt}

Selection (e.g., users clicking or lassoing visual items) is a fundamental
operation in user interfaces and has been well-studied in data visualization.
For example, in Snap-Together Visualization~\cite{north:snap}, multiple views
are coordinated via ``primary-'' and ``foreign-key actions,'' which propagate
selected data tuples from one view to others. Wilhelm~\cite{wilhelm:interaction}
describes the need for such ``indirect object manipulation'' methods as an axiom
of interactive data displays. Chen's compound brushing~\cite{chen:compound}
provides a visual dataflow language for specifying a rich space of
transformations of brush selections.  More recently, Brunel~\cite{brunel}
provides a special \texttt{\#selection} data field that is dynamically populated
with the elements a user interacts with, and can be used to link multiple views
or filter input data. Similarly, RStudio's Shiny~\cite{shiny}, an imperative web
application layer, provides \texttt{brushedPoints} and \texttt{nearestPoints}
functions which can be used to manipulate selected elements.

Other systems have studied formally representing selections as data
queries~\cite{wilhelm:interaction}. For example, brushing interactions in
VQE~\cite{derthick:interactive} generate \emph{extensional} queries that
enumerate all items of interest; a form-based interface enables specification of
\emph{intensional} (declarative) queries. Individual point and brush selections
in DEVise~\cite{livny:devise}, known as \emph{visual queries}, map to a
declarative structure and are used to link together multiple views. Rectangular
``rubber band'' selections in VIQING~\cite{olsten:viqing} are modeled as range
extents, and views can be dropped on top of each other to join their underlying
datasets. Heer et al.~\cite{heer:generalized} demonstrate that declarative
selections can be interactively ``relaxed'' to capture more items of interest.

Reactive Vega and Vega-Lite both build on this work with abstractions for
constructing selections. In Reactive Vega, \emph{predicates} express interactive
selections in visual space, are lifted to data queries by inverting them through
scale functions, and participate in if-then-else chains to manipulate visual
encoding rules. Vega-Lite abstracts these primitives with a higher-level
\emph{selection}. Vega-Lite selections are populated with one or more points of
interest, in response to user interaction. Their built-in predicates map to
declarative queries, thereby allowing a minimal set of ``backing'' points to
represent the full space of selected points. Additional operators transform a
selection's predicate or backing points (e.g., offsetting them to translate a
brush or perform panning). Selections parameterize visual encodings by driving
conditional logic, serving as input data, or defining scale extents.

\vspace{-10pt}

\section{Alternate Paradigms to Imperative Callbacks}

\vspace{-10pt}

For custom interaction design with existing visualization toolkits, users must
typically author imperative event handling callbacks which present several
development hurdles~\cite{myers:callbacks}. Callbacks are registered outside the
visual encoding specification, breaking encapsulation and necessitating
side-effects~\cite{cooper:integrating}. Users are forced to manipulate the
visualization state externally, and manually update
dependencies~\cite{cooper:embedding}. Moreover, callback execution order can be
unpredictable, requiring users to coordinate interleaved
calls~\cite{edwards:coherent}. As a result, callbacks make it difficult to
reason about the current state of the visualization, which is now the product of
both the original specification and the side-effects of interaction callbacks.

\vspace{-10pt}

\subsection{Constraint Programming}

\vspace{-7pt}

One alternative to callback-oriented imperative programming is constraint-based
specification. Systems like Gilt~\cite{myers:callbacks} and
$\mu$constraints~\cite{hudson:constraints} implement one-way constraints: when
the value of a constrained interface is modified, its dependents are
automatically affected. Other systems, like Cassowary~\cite{badros:cassowary},
implement two-way constraints using more complex solvers. Recently,
ConstraintJS~\cite{oney:constraintjs} and Bret Victor's prototype
system~\cite{victor:drawing} have shown that web applications and
visualizations, respectively, can be authored with constraints. However,
constraint systems do not involve general consideration of event
handling\,---\,a crucial element for interaction design. In fact, the authors of
ConstraintJS intend for their system to complement event
architectures~\cite{oney:constraintjs}.

\vspace{-10pt}

\subsection{Functional Reactive Programming}

\vspace{-7pt}

Functional Reactive Programming (FRP) formalizes semantics similar to one-way
constraints. Drawing from dataflow programming, FRP recasts mutable values as
continuous, time-varying data streams~\cite{bainomugisha:frpsurvey}. We focus on
a discrete variant called Event-Driven FRP (E-FRP)~\cite{wan:efrp}. To capture
value changes as they occur, E-FRP provides \emph{streams}, which are infinite
time-ordered sequences of discrete events. Streams can be composed into
\emph{signals} to build expressions that react to events. The E-FRP runtime
constructs the necessary dataflow graph such that, when a new event fires, it
propagates to corresponding streams. Dependent signals are evaluated in two
phases: signals first use the prior computed values of their dependents, which
are subsequently updated in the second phase. E-FRP has been shown to be viable
for authoring web applications~\cite{czaplicki:elm, meyerovich:flapjax} and
visualizations~\cite{cottam:stencil, kelleher:modeljs}.

Reactive Vega's declarative interaction primitives remain grounded in E-FRP
semantics, and they preserve the two-phase update: interdependent signals are
updated in the order in which they are defined in the specification. However,
naive applications of E-FRP to visualization tasks can result in wasteful
recomputation. Traditional E-FRP primitives support only \emph{scalar} values,
whereas visualization pipelines must also process relational and hierarchical
data. Modeling these latter data types as scalar values provides insufficient
granularity to perform targeted recomputation. To efficiently support relational
data, Reactive Vega integrates streaming database methods; and, to support
streaming hierarchical data, the dataflow graph dynamically rewrites itself at
runtime, instantiating new branches to process nested relations.

\vspace{-10pt}

\section{Data Stream Management}

\vspace{-10pt}

The problem of managing streaming data has been well studied in the database
community. Researchers have developed an arsenal of techniques through systems
such as Aurora~\cite{abadi:aurora}, Eddies~\cite{avnur:eddies},
STREAM~\cite{arasu:stream}, and TelegraphCQ~\cite{chandrasekaran:telegraphcq}.
As tuples are observed by these systems, they are flagged as new or removed.
Tuples, rather than full relations, are passed between operators in a query plan
(realized as a dataflow graph). As a result, operators inspect just the updated
tuples to perform efficient computation. However, for some operations a set of
changed tuples is insufficient. For example, a join of two relations requires
access to all tuples within a specified window. In such cases, caches (sometimes
called \emph{views}~\cite{abadi:aurora} or \emph{synopses}~\cite{arasu:stream})
are used to materialize a relation, and shared among dependents.

Borealis~\cite{abadi:borealis} extends this work in two ways. To support
streaming tuple modifications, it introduces a revision processing scheme. An
operator can be \emph{replayed} with revised tuples in place of the original
data, and will then only emit corresponding revisions. Similarly, to enable
dynamic operator parameters, Borealis introduces \emph{time travel}. When an
operator parameter changes, an \texttt{undo} is issued to the nearest cache. The
cache emits tuple deletions, effectively ``rewinding'' the system to a previous
time. A subsequent \texttt{replay} triggers recomputation with the new
parameter.

However, existing streaming data systems concern flat relations. Reactive Vega
instantiates these techniques, alongside E-FRP, within a visualization pipeline
and extends them to support streaming nested data. To do so, Reactive Vega's
dataflow graph dynamically rewrites itself at runtime with new branches. These
branches unpack nested relations, enabling downstream operators to remain
agnostic to higher-level structure while supporting arbitrary levels of nesting.

\vspace{-20pt}

\section{Generative Models of User Interfaces}

\vspace{-15pt}

Michel Beaudoin-Lafon's instrumental interaction
paradigm~\cite{beaudouin:instrumental} helped motivate this dissertation's
approach of considering declarative interaction design at varying levels of
abstraction. He identifies three properties of interaction models for post-WIMP
interaction: \emph{descriptive} of existing and new techniques;
\emph{comparative} to evaluate alternative designs; and, \emph{generative}, to
facilitate the identification and creation of new points in the design space.
Reactive Vega only offers the first property\,---\,an expressive design
space\,---\,but is too low-level to be generative or facilitate evaluation of
similar behaviors. Vega-Lite, on the other hand, offers both an expressive and
generative design space, suitable for exploring alternative techniques as
demonstrated in~\Cref{sec:vl:examples}. Kim et al. also demonstrate its
comparative powers, using Vega-Lite to model optimal sequences of
visualizations~\cite{kim:graphscape}.

Moreover, our articulation of Vega-Lite as a parametric design language is
inspired by Mackinlay's APT~\cite{mackinlay:apt} and his follow-on work with
Card and Robertson conducting a morphological analysis of input
devices~\cite{card:morphological}. In particular, we sought to identify and
decouple orthogonal axes of the design space. While \Cref{sec:vl:examples}
begins to systematically consider points along these axes, a full morphological
analysis of Vega-Lite is left to future work.