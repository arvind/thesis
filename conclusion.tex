% !TEX root = ./thesis.tex
\chapter{Conclusion}
\label{sec:conclusion}

This dissertation addresses the lack of support existing declarative
visualization systems provide for specifying interactive behaviors\,---\,a
critical component of effective
visualization~\cite{pike:interactionscience,yi:understanding}. Through two new
languages, Reactive Vega and Vega-Lite, this dissertation demonstrates that
declarative design is suitably expressive to cover an established taxonomy of
interaction methods for data visualization~\cite{yi:understanding}. Moreover, it
illustrates how declarative specification facilitates retargeting interaction
techniques across multiple modalities (e.g., mouse and touch) and how the
underlying runtime architecture yields interactive performance that meets or
exceeds the current state-of-the-art.

Critically, these languages are at two distinct points in the ladder of
abstraction: Reactive Vega is a lower-level language while Vega-Lite is a
higher-level grammar. As a result, this dissertation also studies the impact of
trading Reactive Vega's expressivity for gains in concision with Vega-Lite.
Namely, Vega-Lite decomposes interaction design into semantic units that can be
systematically enumerated. As a result, not only can users rapidly explore
alternative designs, but higher-level reasoning tasks (e.g., visualization
recommendation~\cite{compassql}) now become more tractable.

These languages offer \textsc{json} syntaxes to facilitate programmatic
generation of visualization, setting the stage for a new breed of interactive
data systems. This dissertation explores this nascent application space with
Lyra, an interactive visualization design environment (VDE). With Lyra, users
are able to author a diverse range of visualizations without writing a single
line of code. As a result, in formative user studies, participants reported that
Lyra \emph{``[made them] feel more in control''} and that \emph{``there's a real
joy in using Lyra.''}

\section{Future Directions in Interactive Data Systems}

\vspace{-10pt}

By enabling programmatic generation of interactive visualizations, the Reactive
Vega stack provides both a platform for developing novel interactive data
systems and, critically, a growing and engaged community of users to study.
Already, using Reactive Vega and Vega-Lite, researchers have built the Voyager
visualization recommendation browser~\cite{voyager,voyager2,compassql}, modeled
optimal sequences of visualizations~\cite{kim:graphscape}, and
reverse-engineered visualizations from chart images~\cite{poco:reverse}.
Moreover, the two languages are used on Wikipedia~\cite{mediawiki:graph}, to
embed interactive visualizations in articles, and in Jupyter
Notebooks~\cite{vega-lite:altair} respectively.

\vspace{-10pt}

\subsection{Automated Design \& Inference over Interactive Visualizations}

\vspace{-7pt}

These systems are only an initial exploration of the space of higher-level
interactive data systems, and there is a fertile ground to study how inference
procedures can be used to accelerate analysis and design. For example, an
immediate next step is to study how Lyra can be extended to support interaction
design by direct manipulation. Analogous to Lyra's existing drag-and-drop data
bindings and scale inference, perhaps users could demonstrate an interactive
behavior with Lyra synthesizing the necessary Vega or Vega-Lite constructs.

Longer-term, Reactive Vega and Vega-Lite provide representations that are
readily amenable for design mining techniques~\cite{kumar:webzeitgeist}. Corpora
of interactive visualization designs can be assembled via
reverse-engineering~\cite{poco:reverse} or
deconstruction~\cite{harper:deconstructing,harper:templates}, and then mined to
codify best practices and identify design trends. This information could then
enable ``auto-complete'' design suggestions, or improve designs with a
visualization ``linter.'' Once deployed, such systems could use \emph{online}
design mining algorithms to react to \emph{emergent} trends.

\vspace{-10pt}

\subsection{Evaluating Expressivity and Usability}

\vspace{-7pt}

This dissertation evaluated the expressiveness of the Reactive Vega stack
through example interactive visualizations. A more rigorous morphological
analysis~\cite{card:morphological} however would not only identify new points in
the design space (e.g., panning color legends as described
in~\secref{sec:vl:panzoom}) but highlight opportunities to refine the models.
For example, while Vega-Lite's visual encoding model can be extended to support
cartographic projections in a straightforward fashion, how to do so to support
pie charts and graph layouts is less clear.

Similarly, while the Cognitive Dimensions of Notation~\cite{blackwell:cogdim}
provide a useful set of heuristic to bootstrap evaluating the usability of the
systems, more formal and longitudinal user studies are necessary to assess the
cognitive burden these tools impose. Such studies can also lead to the
development of new scaffolding to support users. Initial work here has been
promising: through studies with novice Reactive Vega users, Hoffswell
et~al.~\cite{hoffswell:debugging} validated the need to expose system state, and
developed a ``time-traveling'' debugger.

\vspace{-10pt}

\subsection{A Science of Interaction}

\vspace{-7pt}

Developing a generalized theory of interaction\,---\,one that answers questions
such as what makes an interaction technique more effective than another, or what
are principles for combining multiple techniques that preserve their individual
advantages\,---\,has been difficult because existing empirical evaluations of
interactions have been conducted largely in an ad-hoc manner. This is due, in
part, to representations of interaction that have obscured how to isolate
properties of a behavior as experimental variables. As Herbert Simon notes,
\emph{``solving a problem simply means representing it so as to make the
solution transparent''}~\cite{simon:designscience}.

The Reactive Vega stack offers a promising way forward. For a constant Vega-Lite
visual encoding, we can not only systematically generate interaction techniques,
but also vary their constituent properties. These alternative designs could then
be classified using a taxonomy of analytic tasks~\cite{brehmer:taxonomy} and
tested with human subjects. The results will inform our understanding of the
costs and benefits of interactive methods\,---\,for example, are specific
interactive formulations better suited for particular tasks\,---\,and spur the
development of design guidelines, much as graphical perception studies have done
for visual encodings.