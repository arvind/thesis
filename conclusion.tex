% !TEX root = ./thesis.tex
\chapter{Conclusion}
\label{sec:conclusion}

Existing systems have popularized \emph{declarative} visualization design\,---\,
users describe \emph{what} a visualization should look like rather than
\emph{how} it should be computed and rendered. This decoupling of specification
and execution facilitates a more iterative design process, as users can focus on
design decisions while deferring implementation concerns to the system runtime.
However, existing systems provide poor support for interaction, despite it being
central to effective
visualization~\cite{pike:interactionscience,yi:understanding}. Palettes of
common techniques are sometimes offered but limit customization and composition,
and users are forced to program complex event handling callbacks for custom
behaviors. As a result, users must either invest significant effort in
interaction design, or forgo it and risk missing critical insights. Moreover,
existing declarative visualization models are typically embedded within
programming languages. This approach exhibits a poor \emph{closeness of
mapping}~\cite{blackwell:cogdim} as users must map \emph{textual} commands to
their desired \emph{visual} output.

This dissertation addresses these issues through two new declarative
visualization languages, Reactive Vega and Vega-Lite, and a new interactive
design system, Lyra. Reactive Vega reifies four key insights: modeling user
interaction (e.g., mouse clicks or touch gestures) as composable data streams;
constructing named reactive expressions, called signals, from these streams;
lifting signals from the pixel level to the data domain, to coordinate multiple
visualizations; and, finally, endowing existing visual encoding primitives with
reactive semantics. This approach is suitably expressive to cover an established
taxonomy of interaction methods for data visualization~\cite{yi:understanding}
and extends the benefits of declarative specification to interaction design as
well. Users need only describe how input events map to interactive state
changes. The Reactive Vega runtime constructs the requisite dataflow graph and
transparently optimizes processing to yield performance that meets or exceeds
the current state of the art. And, by decoupling input events from downstream
logic, signals simplify retargeting an interaction technique to alternate
modalities.

Vega-Lite moves a step higher in the ladder of abstraction. Whereas Reactive
Vega offers separate primitives for capturing events, building reactive
expressions, and constructing data queries, Vega-Lite encapsulates all this
functionality within a single abstraction called a selection, and offers
extensible selection transformations for further customization. As a result,
Vega-Lite specifications are at least an order of magnitude more concise than
their Reactive Vega counterparts. Moreover, selections and transformations
decompose interaction design into semantic units that can be systematically
enumerated. Thus, not only can users rapidly explore alternative designs, but
higher-level reasoning tasks (e.g., visualization
recommendation~\cite{compassql}) are now more tractable as well.

Critically, Reactive Vega and Vega-Lite are instantiated through \textsc{json}
syntaxes to facilitate programmatic generation of interactive visualizations in
novel interactive data systems. This dissertation explores this higher-level
application space with Lyra, an interactive visualization design environment
(VDE). Users bind data values to mark properties through drag-and-drop
operations, with each such interaction generating a statement in either
Vega-Lite or Reactive Vega. Users are able to author a diverse range of
visualizations without writing a single line of code and, during formative
studies, reported \emph{`` [feeling] more in control''} of the design process
and \emph{``a real joy in using Lyra''}.

\vspace{-10pt}

\section{Future Directions in Interactive Data Systems}

\vspace{-10pt}

The Reactive Vega stack has been released as open-source software, and has seen
widespread adoption among the visualization and data science
communities\,---\,Vega-Lite can be used within a Jupyter
notebook~\cite{vega-lite:altair} and Reactive Vega visualizations can be
embedded within Wikipedia articles~\cite{mediawiki:graph}. As they share the
same underlying foundation, these tools form a new \emph{interoperable
ecosystem} of interactive visualization systems. Users can, for instance,
construct an exploratory visualization in a Jupyter notebook, export it to Lyra
via Vega-Lite, add an explanatory annotation layer, and embed the resultant
Reactive Vega specification within a Wikipedia article. Researchers have also
been adding to this ecosystem with the Voyager visualization recommendation
browser~\cite{voyager,voyager2,compassql}, and tools for sequencing
visualizations~\cite{kim:graphscape}, and reverse-engineering them from chart
images~\cite{poco:reverse}.

Thus, rather than a single monolithic system, the Reactive Vega stack
facilitates development of applications targeted towards specific tasks, and
allows users to work at the level of abstraction most suited for the task at
hand.

\vspace{-10pt}

\subsection{Automated Design \& Inference over Interactive Visualizations}

\vspace{-7pt}

These systems are only an initial exploration of the space of higher-level
interactive data systems, and there is a fertile ground to study how inference
procedures can be used to accelerate analysis and design. For example, an
immediate next step is to study how Lyra can be extended to support interaction
design by direct manipulation. Analogous to Lyra's existing drag-and-drop
interactions, perhaps users \emph{demonstrate} a desired interactive behavior. A
new \emph{interaction inference} architecture will need to be developed, akin to
Lyra's existing scale inference procedures, to interpret these demonstrations.
Initially, heuristics may suffice to simply map demonstrations to Vega-Lite
selections; users would still be responsible for parameterizing visual encodings
with the inferred selections. For instance, if a user clicked multiple points
that shared a common data value, Lyra might suggest a \texttt{multi} selection
with a \texttt{project} transform, which could then be dragged-and-dropped over
the \texttt{color} dropzone to establish a linking interaction. Over time, Lyra
would learn from user behavior to improve this inference\,---\,for example,
suggesting commonly used selections after fewer demonstrations, or even
recommending fully formed interaction techniques such as the aforementioned
linking.

Similarly, Reactive Vega and Vega-Lite provide representations that are readily
amenable for design mining techniques~\cite{kumar:webzeitgeist}. Corpora of
interactive visualization designs can be assembled via
reverse-engineering~\cite{poco:reverse} or
deconstruction~\cite{harper:deconstructing,harper:templates}, and then mined to
codify best practices and identify design trends. This information could then
enable ``auto-complete'' design suggestions (e.g., automatically coloring axis
labels based on mark colors in a dual-axis chart), or improve designs with a
visualization ``linter'' (e.g., ensuring bar charts always begin at a zero
baseline). Once deployed, such systems could use \emph{online} design mining
algorithms to surface \emph{emergent} trends.

\vspace{-10pt}

\subsection{Scalable Interactive Visualization}

\vspace{-7pt}

Besides insights on data stream management, the database literature offers
additional methods that can be applied to Reactive Vega's architecture to better
support interactive visualization of large-scale data. Namely, Reactive Vega's
dataflow graph instantiates a query plan that can be optimized\,---\,a
well-studied subject in database systems~\cite{graefe:volcano,selinger:access}.
With visualization workloads, it is likely that queries issued as a result of
user interaction share common structure or can reuse previously calculated
results. Thus, techniques in multi-query optimization~\cite{roy:multiquery}
(e.g., rewriting queries using views~\cite{halevy:answering}) and adaptive query
processing~\cite{deshpande:adaptive,avnur:eddies} seem most promising.

Moreover, a key property of the Reactive Vega stack is that its abstraction
models are \emph{layered}: high-level Vega-Lite specifications map to low-level
dataflow computational units. As a result, new architectures, such as the one
proposed by Moritz et~al.~\cite{domoritz:dsia}, can be developed to dynamically
partition computation between a client and server, enabling low-latency
interactive visualization of large-scale data. For example, by reasoning about
Vega-Lite semantics, such an architecture could automatically reuse data already
loaded on the client for a zooming interaction, prefetch the previous/next
batches of data for panning operations, or adapt the resolution of server-side
aggregation when the viewport size is changed.

\vspace{-10pt}

\subsection{Evaluating Expressivity and Usability}

\vspace{-7pt}

This dissertation evaluated the expressiveness of the Reactive Vega stack
through example interactive visualizations. A more rigorous analysis of the
design space, like the morphological analysis of input devices conducted by Card
et~al.~\cite{card:morphological}, would identify new points (e.g., panning color
legends as described in~\secref{sec:vl:panzoom}) and highlight opportunities to
further refine the models. For example, common statistical displays such as
error bars and box plots can now be constructed by composing several Vega-Lite
unit specifications. To simplify these common use cases, one could extend
Vega-Lite's existing mark types with new \emph{composite} types that function
like macros. However, the semantics of selections against composite mark types
remains an open question. For instance, what does a single selection of a box
plot mean? Is the whole box plot itself selected, or the underlying aggregated
values? How might we specify selecting constituent components of the box plot,
(e.g., a specific quartile or whisker) when they cannot be addressed with the
project transform?

Similarly, while the Cognitive Dimensions of Notation~\cite{blackwell:cogdim}
provide a useful set of heuristic to bootstrap evaluating the usability of the
systems, more formal and longitudinal user studies are necessary to assess the
cognitive burden these tools impose. Are new users able to learn the declarative
interaction design model? Can expert users, long-accustomed to event handling
callbacks, adapt quickly? What pitfalls do the two user classes experience? Such
studies can also lead to the development of new scaffolding to support users.
For instance, the Reactive Vega dataflow graph offers an execution model that
can be readily visualized, and new instrumentation to enable inspection and
stepping through changeset propagation could aid learnability~\cite{guo:tutor}.
Initial work here has been promising: a ``time-traveling'' debugger, developed
by Hoffswell et al.~\cite{hoffswell:debugging}, was found to help first-time
Vega users accurately identify the source of errors, with some event attempting
to fix them.

\vspace{-10pt}

\subsection{A Science of Interaction}

\vspace{-7pt}

Developing a generalized theory of interaction\,---\,one that answers questions
such as what makes an interaction technique more effective than another, or what
are principles for combining multiple techniques that preserve their individual
advantages\,---\,has been difficult because existing empirical evaluations of
interactions have been conducted largely in an ad-hoc manner. This is due, in
part, to representations of interaction that have obscured how to isolate
properties of a behavior as experimental variables.

The Reactive Vega stack offers a promising way forward. For a constant Vega-Lite
visual encoding, we can systematically generate interaction techniques and vary
their constituent properties. These alternative designs could then be classified
using a taxonomy of analytic tasks~\cite{brehmer:taxonomy} and tested with human
subjects. The results will inform our understanding of the costs and benefits of
interactive methods\,---\,for example, are specific interactive formulations
better suited for particular tasks\,---\,and spur the development of design
guidelines, much as graphical perception studies have done for visual encodings.

\vspace{-10pt}

\section{Concluding Remarks}

\vspace{-10pt}

\setlength\epigraphwidth{.8\textwidth}
\epigraph{\textit{Solving a problem simply means representing it so as to make
the solution transparent.}} {\textit{The Sciences of the
Artificial}\\\textsc{Herbert Simon, 1981}}

The lack of unified support for interaction design in declarative visualization
systems has not only imposed an artificially high technical burden on end-users,
but it has also hindered researchers' ability to study interaction as a
first-class citizen of visualization design, and develop interactive data
systems. This dissertation charts a way forward with the Reactive Vega stack,
available as open-source software at \url{http://vega.github.io/}.