% !TEX root = ../thesis.tex
\section{Discussion: Cognitive Dimensions of Notation}
\label{sec:vg:discussion}

The previous section demonstrated the expressivity of Reactive Vega's model of
declarative interaction design. Here, we evaluate it from the perspective of a
visualization designer using the Cognitive Dimensions of
Notation~\cite{blackwell:cogdim}, a set of heuristics for evaluating the
efficacy of notational systems such as programming languages and interfaces. Of
the 14 dimensions, we evaluate Reactive Vega against a relevant subset and
compare it against current common practice: declarative specification of visual
encodings using D3~\cite{bostock:d3} and imperative event handling callbacks for
interaction.

\textbf{Abstractions} (types and availability of abstraction mechanisms) and
\textbf{Viscosity} (resistance to change). Streams and signals abstract the
low-level events that trigger interactions and decouple them from downstream
logic. This approach can facilitate rapid iteration: the result of an
interaction can be designed (for example, highlighting points), and then a
variety of different event triggers can be prototyped by simply rebinding the
appropriate signals. As our examples demonstrate, rebinding signals reduces the
burden for resolving conflicting interactions or retargeting to different
platforms. By comparison, iterating with event callbacks can be more difficult.
A particular sequence of events may require a specific ordering of callbacks,
and coordinating the visualization state across these functions falls to the
designer.

\textbf{Premature Commitment} (constraints on the order of doing things).
Abstracting streams into signals does impose a premature commitment. Users must
define them before they are able to use any lower-level events to trigger state
changes. This requirement could be relaxed: users could use event streams
inline, for example within a predicate or production rule. However, we believe
such inline references are a poor design pattern as they make an interaction
technique dependent on a specific set of events\,---\,a common problem with
existing interactor typologies. Moreover, reuse would be hampered as it would
become more difficult to resolve conflicting interactions (e.g., brushing and
panning) or retarget techniques across input modalities.

\textbf{Hidden Dependencies} (important links between entities are not visible).
Signals do introduce hidden dependencies as they obscure which input events
trigger particular changes to the state of the visualization. However, we
believe that the gains in viscosity outweigh the complexities of these hidden
dependencies, particularly as the latter can be ameliorated by naming signals
appropriately. Furthermore, as our code examples illustrate, all the factors
that directly affect a particular state are captured within a single Reactive
Vega specification. For example, a signal definition specifies all events or
signals that may affect its value; similarly, a visual property may use a rule
which enumerates all the values it may take. With D3, the visual encoding
specification may not completely define all states. Instead, the user must trace
a flow through event callbacks, a process further exacerbated by an
unpredictable execution order. The user is forced to coordinate interleaved
callbacks, introducing \textbf{hard mental operations} and
\textbf{error-proneness}.

\textbf{Consistency} (similar semantics are expressed in similar syntactic
forms). Our interaction model is best suited for systems that declaratively
specify visual encodings, and would feel foreign in imperative systems. However,
given the widespread adoption of D3, and Vega's increasing integration in
systems~\cite{lyra, voyager, wikipedia:graph} we believe this is not a
significant liability. By contrast, registering event callbacks on D3
visualizations breaks consistency: visual design is declarative while
interaction design is imperative. It requires users to think in terms of
different notational systems, and exposes underlying implementation and
execution concerns.

\textbf{Visibility} (ability to view components easily). One of the primary
advantages of using D3, and registering event callbacks, is being able to debug
code directly within a web browser~\cite{bostock:d3}: the generated
visualization can be inspected, while the JavaScript console can be used to
interactively debug event callbacks. Reusing such existing scaffolding is more
difficult with Reactive Vega as its runtime, which parses and renders a
specification, introduces its own stack of abstractions. However, we believe
this gap offers a promising avenue for future work in new development
environments that can leverage Vega's reactive semantics. For instance, initial
work by Hoffswell et al.~\cite{hoffswell:debugging} has devised a
``time-travelling'' debugger for Reactive Vega specifications. In particular,
the authors note that signals are a critical abstraction barrier, providing a
meaningful entry-point into an interaction specification. To construct a similar
debugger for imperative event handling callbacks would require complex static
analysis to identify and surface relevant program state.

In summary, Reactive Vega introduces hidden dependencies between state changes
and their triggering input events and, without additional tooling, decreases
visibility into runtime execution. However, we believe these drawbacks are
outweighed by the increase in specification consistency between visual encodings
and interaction, and a decrease in viscosity, allowing users to iterate more
quickly over interaction design.