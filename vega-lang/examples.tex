% !TEX root = ./thesis.tex
\section{Example Interactive Visualizations}
\label{sec:vg:examples}

To evaluate the expressivity of our language, we present a range of examples and
demonstrate coverage over Yi~et~al.'s interaction
taxonomy~\cite{yi:understanding}. Yi~et~al. identify seven categories based on
user intent: \emph{select}, to mark items of interest; \emph{connect}, to show
related items; \emph{abstract/elaborate}, to show more or less detail;
\emph{explore}, to examine a different subset of data; \emph{reconfigure}, to
show a different arrangement of data; \emph{filter}, to show something
conditionally; and, \emph{encode}, to use a different visual encoding. It is
important to note that these categories are not mutually exclusive, and an
interaction technique can be classified under several categories. We choose
example interactive visualizations to demonstrate that our model can express
interactions across all seven categories and how, through composition of its
primitives, supports the accretive design of richer interactions.

\subsection{Selection: Click/Shift-Click and Brushing}

\cref{}\todo{Figure} provides a snippet of Reactive Vega JSON to highlight
points that a user clicks. A signal constructed over a click stream feeds a data
transform that toggles values in a data source (named \texttt{selected\_pts}).
An intensional predicate test whether the shift key is pressed and, if not,
clears the data source prior to inserting the clicked values. An extensional
predicate is used within a production rule to set the fill colour selected
points.

Similarly, \cref{}\todo{Figure} demonstrates the Reactive Vega JSON necessary to
enable brush selections. Signals are registered to capture the start and end
positions of the brush, by default \texttt{mousedown} and \texttt{[mousedown,
mouseup] > mousemove}, respectively. Scale inversions are invoked to calculate
the data extents of the brush, which are used to define an intensional predicate
to express the brushed data range. As before, the predicate is used within a
production rule to set the fill colour of selected points.

\subsection{Connect: Brushing \& Linking}

We can extract the interaction from the previous example into a standalone
``brushing'' interactor, and then apply it to brush \& link a scatterplot matrix
as shown in \cref{}\todo{SPLOM}. Each cell of the matrix is an instance of a
group mark with its own coordinate space. The plotting symbol and necessary
spatial scale functions are defined within this group. Had the interactor's
predicates defined selections over pixel space, the production rule would
highlight points that fall along the same horizontal and vertical pixel
regions\,---\,as shown in \cref{}\todo{}, brushing over orange
(\texttt{versicolor}) points would also highlight red (\texttt{virginica}) and
blue (\texttt{setosa}) points. Instead, the interactor uses scale inversions to
lift the predicate to the data domain. Thus, the production rule correctly
performs the linking operation across scatterplots.

\subsection{Abstract/Elaborate: Overview\,+\,Detail}

With our brush interactor, we can also create the overview + detail
visualization shown in Figure~\ref{fig:range_predicate}. In this case, brushing
is restricted to the horizontal dimension. In our visualization, we override the
\texttt{height} property of the visual brush added by the interactor, and ignore
the vertical range predicates it populates. We use the horizontal range
predicate with a filter transformation, to filter points for display in the
detail plot. As a user draws a brush, signals update the horizontal range
predicate, which in turn reactively filters points in the data source, updates
scale functions and re-renders the detail view.

\subsection{Explore \& Encode: Panning \& Zooming}

\subsection{Reconfigure: Index Chart}

\subsection{Filter: Control Widgets}

