% !TEX root = ./thesis.tex
\chapter{Introduction}

\vspace{-35pt}

\setlength\epigraphwidth{.75\textwidth}
\epigraph{\textit{A graphic is not ``drawn'' once and for all; it is
``constructed'' and reconstructed until it reveals all the relationships
 constituted by the interplay of the data. The best graphic operations are those
 carried out by the decision-maker himself.}} {\textit{Graphics and Graphic
 Information Processing}\\\textsc{Jacques Bertin, 1981}}

Data visualization has gone mainstream. From business intelligence to data
journalism, society has embraced visualization as a medium for recording,
analyzing, and communicating about data. This explosion of use is reflected in
the proliferation of visualization tools, which span the gamut of expressivity.
At one end are chart templates (e.g., as found in spreadsheet packages such as
Microsoft Excel). Users can easily generate recognizable output by choosing from
a pre-defined palette of chart types, but only a handful of properties may be
customized. At the other end are vector graphics programs (e.g., Adobe
Illustrator), and low-level programming APIs (e.g., Processing or OpenGL). Users
have complete control but the added expressivity trades-off ease-of-use.

Advances in visualization toolkits have popularized \emph{grammar-based}
approaches for visual encodings
\cite{wilkinson:grammar,stolte:polaris,wickham:layered,bostock:protovis,bostock:d3}.
These models are more expressive than chart templates as they decompose the
design space into combinatorial primitives for data transformation and visual
encoding. Moreover, they embody \emph{declarative design}\,---\,the languages
describe \emph{what} the visualization should look like, rather than \emph{how}
it should be rendered. As a result, users are free to focus on visual encoding
and design decisions while the underlying runtime is responsible for execution
concerns and performance optimization~\cite{heer:protovisjava}.

However, key challenges remain. Interaction is critical for effective
visualization, allowing users to interrogate data and iteratively refine their
mental models~\cite{pike:interactionscience,yi:understanding}. Yet, existing
declarative visualization models provide poor support for interaction
techniques. Simple ``interaction typologies'' (e.g., brushing, panning, etc.)
are available, but limit customization. For custom interaction, users must
author \emph{imperative} event handling callbacks that undo the benefits of
declarative design. Users are forced to manually maintain
state~\cite{cooper:embedding} and coordinate interleaved execution\,---\,a
complex and error-prone task colloquially called ``callback
hell''~\cite{edwards:coherent}. As a result, users must either invest
significant effort in interaction design, or choose to forgo it and risk missing
critical insights.

Moreover, most existing declarative models (including
ggplot2~\cite{wickham:layered} and D3~\cite{bostock:d3}) are instantiated via
complex APIs embedded in programming languages. This approach imposes a
non-trivial \emph{articulatory distance}~\cite{hutchins:directmanip} as users
must map their desired \emph{visual} output to \emph{textual} commands\,---\,a
fundamental mismatch in representations.

\vspace{-10pt}

\section{Thesis Contributions and Outline}

\vspace{-10pt}

This dissertation is divided into seven chapters, and contributes a layered
stack of new declarative languages for \emph{interactive} visualization as well
as a new interactive system for visualization design through direct
manipulation.

\Cref{sec:related_work} surveys prior work that this dissertation builds on,
including a panoply of visualization toolkits and systems, methods from
functional programming languages and streaming database systems, and generative
models of user interfaces.

\Cref{sec:vg:lang} introduces Reactive Vega: a declarative language for
interactive data visualization that models user interaction as \emph{streaming
data}. Alongside established declarative visual encoding primitives, Reactive
Vega introduces event streams and signals, two constructs from Functional
Reactive Programming. Event streams, defined with a novel CSS-inspired selector
syntax, abstract the complexity of capturing and sequencing input events. Event
streams drive signals: dynamic expressions that are automatically reevaluated
when new events fire. Signals parameterize visual encoding primitives, thereby
endowing them with reactive semantics\,---\,when a new event fires, it is
propagated to corresponding signals; dependent visual encodings are reevaluated
and the visualization is automatically re-rendered. Additional primitives allow
interaction techniques to be generalized and reused across distinct
visualizations. This chapter demonstrates Reactive Vega's expressivity through
example visualizations that cover an established interaction taxonomy.

\Cref{sec:vg:arch} studies the implications of declarative interaction design on
the architecture of visualization systems. The Reactive Vega architecture adapts
and extends techniques from streaming database systems. In particular, an input
declarative specification is parsed to construct a dataflow graph. Dataflow
operators perform computations on \emph{changesets} of tuples or scene graph
elements, and are \emph{replayed} if signal parameter values change. To support
data-driven multi-view displays, the dataflow graph is conditioned on data
values. As new tuples are observed, or as interaction events occur, the graph
\emph{dynamically rewrites itself} at runtime by extending or pruning branches.
Performance optimizations are described and comparative benchmark studies
conducted to evaluate Reactive Vega's performance. These studies find that
Reactive Vega meets or exceeds the performance of D3~\cite{bostock:d3}, the
current state-of-the-art system.

Reactive Vega's primitives, however, are relatively low-level. They yield
verbose specifications that impede a rapid authoring process and hinder
exploration of alternative designs. \Cref{sec:vega-lite} introduces Vega-Lite: a
higher-level grammar that builds on Reactive Vega. Vega-Lite extends a
traditional grammar of graphics with a \emph{view composition algebra} for
layered and multi-view displays. A novel \emph{grammar of interaction} features
\emph{selections}: an abstraction that encapsulates input event processing,
points of interest, and a predicate function for inclusion testing. Selections
parameterize visual encodings by serving as input data, defining scale extents,
or by driving conditional logic. Vega-Lite specifications are concise through
ambiguity: lower-level details (e.g., the events that trigger a selection) can
be omitted and are filled in by the Vega-Lite compiler. The smaller language
surface area also facilitates higher-level reasoning tasks, such as systematic
enumeration of alternative designs. By recreating the examples from
\cref{sec:vg:lang}, this chapter demonstrates how Vega-Lite enables succinct
specification and broader exploration of the design space.

Reactive Vega and Vega-Lite provide \textsc{json} syntaxes to facilitate the
programmatic generation of visualizations in novel higher-level interactive data
systems. \Cref{sec:lyra} describes one such system, Lyra, that enables direct
manipulation visualization design. Through drag-and-drop interactions, users
bind data values to mark properties. Data transformations and layout algorithms
are visually specified and inspected with a ``data pipeline'' interface.
Critically, Lyra allows users to fluidly move between the two levels of
abstraction. Direct manipulation operations generate Vega-Lite statements, which
are compiled, and merged into a backing Reactive Vega specification; the visual
inspectors provide complete control over the latter. Thus, users can iterate
between rapidly creating recognizable output and making fine-grained
customizations\,---\,an approach that yields diverse range of visualizations
without writing a single line of code.

Finally, \Cref{sec:conclusion} sketches new research directions that Reactive
Vega and Vega-Lite enable. These systems have been released as open-source
projects, and have already been used to power the Voyager visualization
recommendation browser~\cite{voyager,voyager2,compassql}, develop a model for
sequencing visualizations~\cite{kim:graphscape}, and reverse-engineer
visualizations from chart images~\cite{poco:reverse}. Moreover, the broader
adoption both tools have seen\,---\,Vega is used to embed interactive
visualizations within Wikipedia articles, and Vega-Lite can be used within
Jupyter notebooks\,---\,provide a growing, engaged user community to study their
use with.

\vspace{-20pt}

\section{Prior Publications and Authorship}

\vspace{-10pt}

Although I am the principal author of the research detailed in this
dissertation, it is also the product of years of collaboration with my primary
advisor, Jeffrey Heer, and my colleagues at the University of Washington
Interactive Data Lab. The Reactive Vega language (\Cref{sec:vg:lang}) appeared
at \textsc{acm uist} 2014~\cite{reactive-vega-model} with Kanit
Wongsuphasawat contributing a prototype implementation of the reactive runtime.
The Reactive Vega architecture (\Cref{sec:vg:arch}) was published at
\textsc{ieee vis} 2015~\cite{reactive-vega-arch}, with Ryan Russell and Jane
Hoffswell contributing example visualizations and figures. Vega-Lite was
published at \textsc{ieee vis} 2016~\cite{vega-lite}. It is joint work with
Dominik Moritz and Kanit Wongsuphasawat, who led the design and implementation
of the visual encodings. Finally, Lyra was published at EuroVis
2014~\cite{lyra}. To reflect my collaborators' contributions, I will use the
first-person plural in these chapters.